% !TeX encoding = UTF-8
% !TeX root = choix_extensions.tex
\chapter{Listes}
\label{ch:gestionListes}

Pour gérer finement les listes, rien ne vaut le package \href{http://mirror.ctan.org/macros/latex/contrib/enumitem/enumitem.pdf}{enumitem}. En l'appelant avec l'option \mintinline{latex}{shortlabels} \newline (\mintinline{latex}{\usepackage[shortlabels]{enumitem}}), il est compatible avec l'ancienne extension \mintinline{latex}{enumerate} qui est largement moins puissant.





\section[enumerate]{Énumérations spéciales}



\subsection*{Personnalisation de la numérotation}

\begin{LTXexample}[pos=o]
\begin{enumerate}[a.]
    \item Un
    \item Deux
    \item Trois
\end{enumerate}
\end{LTXexample}

\begin{LTXexample}[pos=o]
\begin{enumerate}[1.]
    \item Un
    \item Deux
    \item Trois
\end{enumerate}
\end{LTXexample}

\begin{LTXexample}[pos=o]
\begin{enumerate}[1)]
    \item Un
    \item Deux
    \item Trois
\end{enumerate}
\end{LTXexample}



\subsection*{Raccourcis}

Le fichier \incmd{preambule_college.sty} contient les commandes suivants : \mintinline{latex}{enuma}, \mintinline{latex}{enum}, \mintinline{latex}{enumpar} :

\begin{LTXexample}[pos=o]
\enuma{
    \item Un
    \item Deux
    \item Trois
}
\end{LTXexample}

\begin{LTXexample}[pos=o]
\enum{
    \item Un
    \item Deux
    \item Trois
}
\end{LTXexample}

\begin{LTXexample}[pos=o]
\enumpar{
    \item Un
    \item Deux
    \item Trois
}
\end{LTXexample}





\section{Régler les espacements}



\subsection{Verticaux}

L'option \mintinline{latex}{itemsep} permet d'aérer la liste et \mintinline{latex}{topsep} règle l'espacement avec le texte avant ou après la liste (il y a encore d'autres paramètres plus pointus dans la documentation d'\mintinline{latex}{enumitem}).

\begin{LTXexample}[pos=o,width=.4]
Texte précédent
\begin{enumerate}[itemsep=.2cm, topsep=.5cm]
    \item Un
    \item Deux
    \item Trois
\end{enumerate}
Texte suivant
\end{LTXexample}



\subsection{Horizontaux}

Pour éviter l'indentation de la liste : \mintinline{latex}{leftmargin=*}. Pour régler l'espacement entre le numéro et le texte : \mintinline{latex}{labelsep=} et la distance voulue.
\begin{LTXexample}[pos=o,width=.4]
Texte précédent
\begin{enumerate}[leftmargin=*, labelsep=1cm]
    \item Un
    \item Deux
    \item Trois
\end{enumerate}
Texte suivant
\end{LTXexample}


Tout aligner horizontalement sur la fin des labels :
\begin{LTXexample}[pos=o,width=.4]
\begin{description}[
    font=\normalfont,
    leftmargin=!,
    labelwidth=\widthof{2000 :}]
    \item[1994 :] Le 1er octobre, le \href{http://www.w3.org}{w3c} est créé pour définir des protocoles \dots
    \item[1995 :] Le web compte 25'000 sites en ligne
\end{description}
\end{LTXexample}





\section{Listes dans un paragraphe spécial}
\label{sec:listeParSpeciaux}

Si un des paragraphes spéciaux des sections \ref{sec:paragrapheSpecialDefin} et \ref{sec:paragraphesSpeciauxThmtools} débute immédiatement pas une liste (sans aucun texte entre le titre du paragraphe spécial et la liste), il manque un retour à la ligne. Le raccourci \mintinline{latex}{\listtopsep} permet de régler le problème :

\begin{LTXexample}[pos=o,width=.4]
\exemples{
      \listtopsep
      \begin{itemize}
             \item Avec
             \item précaution
      \end{itemize}
}
\end{LTXexample}





\newpage





Parfois, il est nécessaire de régler cet espace au cas par cas. On peut le faire :
\begin{enumerate}[a.]
	\item de manière semi-automatique;
		\begin{LTXexample}[pos=o,width=.4]
\exemples{
    \leavevmode
    \vspace*{-\parskip}
    \vspace*{-\baselineskip}
    \begin{itemize}
        \item Avec
        \item précaution
    \end{itemize}
}
		\end{LTXexample}
	
	\item à la main, par exemple ici l'espacement est de \mintinline{latex}{-3ex}.
	   \begin{LTXexample}[pos=o,width=.4]
\exemples{
   \leavevmode
   \vspace*{-3ex}
   \begin{itemize}
       \item Avec
       \item précaution
   \end{itemize}
}
	\end{LTXexample}
\end{enumerate}