% !TeX encoding = UTF-8
% !TeX root = choix_extensions.tex
\chapter{Paragraphes spéciaux}
\label{ch:paragraphesSpeciaux}

Certains paragraphes spéciaux sont très fréquents dans nos documents : \emph{Exemples}, \emph{Définition}, \emph{Remarque}, \emph{Théorèmes}\dots \ Une grosse partie du fichier \incmd{preambule_college.sty} sert à définir des raccourcis pour de tels paragraphes. 




\section{Raccourcis pour les définitions}
\label{sec:paragrapheSpecialDefin}

Les commandes suivantes permettent de gérer les définitions :
\begin{enumerate}
	\item \mintinline{latex}{\defin[titre]{paragraphe}} permet de produire un cadre pour une définition;
	\item \mintinline{latex}{\defins[Titre]{Définitions}}) : la même au pluriel;
	\item \mintinline{latex}{\emphdef} permet de mettre en évidence les mots à définir.
\end{enumerate}

\remarque*{
	On peut modifier la numérotation (voir \ref{sec:numerotationParagraphes}), mais pas la supprimer.
}

\begin{LTXexample}[pos=o,width=.4]
\defin{
    Définition
}
\defins{
    Version au \emphdef{pluriel}
}
\defin[titre]{
    Version avec \emphdef{titre} en option
}
\defins[titre]{
    Les \emphdef{deux variantes}
}
\end{LTXexample}





\section{Raccourcis pour des paragraphes fréquents}
\label{sec:paragraphesSpeciauxThmtools}

Le fichier \incmd{preambule_college.sty} présente toute une série de raccourcis pour les paragraphes les plus courants, chacun dans trois variantes (voir l'exemple ci-dessous):
\begin{enumerate}
	\item un titre en option (\mintinline{latex}{\exemple[Titre]{Exemple}});
	\item une version au singulier et une au pluriel (\mintinline{latex}{\exemple} et \mintinline{latex}{\exemples});
	\item une version normale et une étoilée, donc sans numérotation (\lstinline!\exemple*!)
\end{enumerate}

\erreurCourante*{
	Dans le code \LaTeX, il n'est pas possible de laisser une ligne vide dans un de ces environnements : une erreur de compilation est générée (p. ex. \inlatex{Paragraph ended before \probleme was complete.})
}

Les fichiers \incmd{.sty} proposés contiennent la définition des commandes suivantes :
\begin{itemize}[leftmargin=*]
	\begin{multicols}{4}
		\item \mintinline{latex}{\exemple}
		\item \mintinline{latex}{\contreExemple}
		\item \mintinline{latex}{\remarque}
		\item \mintinline{latex}{\exercice}
		\item \mintinline{latex}{\exerciceEtoile}
		\item \mintinline{latex}{\probleme}
		\item \mintinline{latex}{\problemeEtoile}
		\item \mintinline{latex}{\rappel}
		\item \mintinline{latex}{\rappelEtoile}
		\item \mintinline{latex}{\propriete}
		\item \mintinline{latex}{\notionIntuitive}
		\item \mintinline{latex}{\hypothese}
		\item \mintinline{latex}{\these}
		\item \mintinline{latex}{\conclusion}
		\item \mintinline{latex}{\demonstration}
		\item \mintinline{latex}{\notation}
		\item \mintinline{latex}{\algorithme}
		\item \mintinline{latex}{\erreurCourante}
		\item \mintinline{latex}{\bestPractice}
		\item \mintinline{latex}{\astuce}
	\end{multicols}
\end{itemize}

\begin{LTXexample}[pos=o,width=.4]
\exemple{
    Exemple de base
}
\exemples{
    Exemples (au pluriel)
}
\exemple[titre]{
    Exemple avec titre
}
\exemples[titre]{
    Exemple au pluriel avec titre
}
\exemple*{
    Exemple sans numérotation
}
\exemples*{
    Au pluriel sans numérotation
}
\exemple*[titre]{
    Avec titre, sans numérotation
}
\exemples*[titre]{
    Avec titre, au pluriel, sans numérotation
}
\end{LTXexample}

\remarques*{
	\listtopsep
	\begin{enumerate}
		\item Lorsque ces paragraphes débutent immédiatement par une liste (comme c'est le cas pour ce paragraphe-ci), il manque un retour à la ligne. Pour le forcer, voir \ref{sec:listeParSpeciaux}.
		\item Ces paragraphes sont réalisés avec l'extension \href{http://mirror.ctan.org/macros/latex/exptl/thmtools/thmtools.pdf}{thmtools}. La section \ref{sec:numerotationParagraphes} présente comment adapter la numérotation à ses besoins. Pour le reste, cette extension permet de nombreuses variantes de cadre, d'ombrage, de couleurs\dots
		\item Les commandes , \mintinline{latex}{\problemeEtoile} et \mintinline{latex}{\rappelEtoile} servent à faire afficher une étoile dans le document final, par exemple pour indiquer un exercice plus difficile ou un rappel spécial.
	\end{enumerate}
}





\section{Raccourcis pour des paragraphes avec cadres}
\label{sec:paragraphesSpeciauxCadre}

Certains paragraphes sont mis en forme avec des cadres réalisés en \mintinline{latex}{TikZ}\footnote{TikZ est un langage graphique pour \LaTeX.}. Ces commandes sont disponibles en version étoilée (conformément à la syntaxe \LaTeX, les versions étoilées n'ont pas de numérotation) et prennent un titre en option.

\begin{LTXexample}[pos=o,width=.4]
\axiome{
    Un axiome
}
\axiome*{
    Un axiome
}
\end{LTXexample}


Les commandes suivantes sont disponibles :
\begin{multicols}{3}
	\begin{itemize}
		\item \mintinline{latex}{\axiome}
		\item \mintinline{latex}{\theoreme}
		\item \mintinline{latex}{\corollaire}
		\item \mintinline{latex}{\equivalence}
		\item \mintinline{latex}{\regle}
	\end{itemize}
\end{multicols}





\section[Numérotation]{Numérotation des paragraphes spéciaux et références}
\label{sec:numerotationParagraphes}


\subsection[Type de numérotation]{Modifier le type de numérotation}

On peut modifier facilement la numérotation des paragraphes spéciaux dans le fichier \mintinline{latex}{preambule_college.sty}\footnote{Moins facile, mais plus propre : redéfinir le paragraphe spécial dans \mintinline{latex}{preambule_college.sty}.}. Par exemple, la numérotation des rappels recommence à 1 à chaque chapitre à cause du code suivant :

{\small \Verb!\declaretheorem[name=Rappel,style=collegeBase,!{\color{red} \Verb!parent=chapter!}\Verb!]{rappelThm}!}

Pour éviter que la numérotation ne recommence à zéro à chaque chapitre, il suffit d'enlever \mintinline{latex}{parent=chapter}. On peut évidemment mettre autre chose que \mintinline{latex}{chapter} après \mintinline{latex}{parent}. Pour plus de détails, voir la documentation de \href{http://mirror.ctan.org/macros/latex/exptl/thmtools/thmtools.pdf}{thmtools}.



\subsection{Modification de la numérotation}

Les paragraphes spéciaux sont créés à partir d'environnements \mintinline{latex}{theorem}. Par exemple, pour \mintinline{latex}{\exercice}, on voit, dans \mintinline{latex}{preambule_college.sty}, qu'il y a un environnement \mintinline{latex}{exerciceThm}. A cet environnement est associé un compteur de même nom qu'on peut régler avec \mintinline{latex}{\setcounter}.

Par exemple on crée des exercices et on donne ensuite la solution en faisant repartir à zéro le compteur associé :

\begin{LTXexample}[pos=o,width=0.4]
\setcounter{exerciceThm}{8}

\subsubsection{Données}
    \exercice{Neuvième exercice}
    \exercice{Dixième exercice}

\setcounter{exerciceThm}{8}

\subsubsection{Solutions}
    \exercice{Solution de l'exercice 9}
    \exercice{Solution de l'exercice 10}
\end{LTXexample}





\newpage





\subsection{Références vers des paragraphes}

Les paragraphes spéciaux sont, en principe, tous construits sur le même modèle. Les commandes \mintinline{latex}{\rappel} et \mintinline{latex}{\rappels} permettent d'illustrer le principe.

\begin{minipage}{.65\linewidth}
	\begin{minted}{latex}
\rappel{
    \label{rappel1}
    Premier point de référence.
}
\rappels[Rappel avec un titre]{
    \label{rappel2}
    Deuxième point de référence.
}

Faisons référence :
\begin{enumerate}
    \item au premier rappel (le \autoref{rappel1});
    \item seulement à son numéro (le \ref{rappel1});
    \item au nom du deuxième (\nameref{rappel2}).
\end{enumerate}
	\end{minted}
\end{minipage}
\hfill
\begin{boxedminipage}{.34\linewidth}
	\rappel{
		\label{rappel1}
		Premier point de référence.
	}
	\rappels[Rappel avec un titre]{
		\label{rappel2}
		Deuxième point de référence.
	}
	
	Faisons référence :
	\begin{enumerate}
		\item au premier rappel (le \autoref{rappel1});
		\item seulement à son numéro (le \ref{rappel1});
		\item au nom du deuxième (\nameref{rappel2}).
	\end{enumerate}
\end{boxedminipage}





\section{Problèmes connus}



\subsection{Note de bas de page dans un paragraphe spécial}

Lorsqu'un appel de note de bas de page reste collé à un environnement, par exemple dans une définition ou un paragraphe encadré, il est possible de l'envoyer tout de même en bas de page.

\subsubsection{Note mal placée}

\defin{
	Une définition avec une note de bas de page dans le cadre \footnote{Note dans le cadre}
}

Exemple réalisé avec ce code :

\begin{minted}{latex}
\defin{
  Une définition avec une note de bas de page dans le cadre
  \footnote{Note dans le cadre
}
\end{minted}

\subsubsection{Note placée correctement}

\defin{
	Une définition avec une note en bas de page
	\footnotemark
}
\footnotetext{Note de bas de page au bon endroit}
Exemple réalisé avec ce code :

\begin{minted}{latex}
\defin{
  Une définition avec une note en bas de page
  \footnotemark
}
\footnotetext{Note de bas de page au bon endroit}
\end{minted}



\subsection{Problème de filet dans une définition}

\begin{minipage}[t]{.5\linewidth}
	Il arrive que, lorsque la définition se situe en haut d'une page, le filet horizontal coupe le texte au lieu d'arriver juste en dessous comme dans l'exemple ci-contre.
	
	Dans une telle situation, il suffit de mettre un \inlatex{\newpage} juste avant pour forcer un saut de page propre et tout revient dans l'ordre.
\end{minipage}
\hfill
\begin{minipage}[t]{.45\linewidth}
	\vspace{-5ex}
	\defin{
		Lorem ipsum dolor sit amet, consectetur adipiscing elit. Cras fermentum feugiat nisi in condimentum.
		\vspace*{-2ex}
	}
\end{minipage}



\subsection{Erreur "Paragraph ended before <nom> was complete" à la compilation}

Ce message provient d'une ligne vide dans le code, à l'intérieur d'un paragraphe spécial. Par exemple, une ligne vide pour séparer deux \inlatex{\item} dans une liste produit ce message d'erreur. Utiliser \inlatex{\\} au besoin, mais ne pas introduire de ligne vide !





\section{D'autres personnalisations}



\subsection{Avec tcolorbox}

C'est le dernier package arrivé et sans doute le plus intéressant et le plus puissant ! Les autres peuvent être largement ignorés. Un exemple tiré du manuel (de 428 pages quand même\dots)

\begin{LTXexample}[pos=o]
	\begin{tcolorbox}[title=My title,
		colback=red!5!white,
		colframe=red!75!black,
		colbacktitle=yellow!50!red,
		coltitle=red!25!black,
		fonttitle=\bfseries,
		subtitle style={boxrule=0.4pt,
			colback=yellow!50!red!25!white} ]
		This is a \textbf{tcolorbox}.
		\tcbsubtitle{My subtitle}
		Further text.
		\tcbsubtitle{Second subtitle}
		Further text.
	\end{tcolorbox}
\end{LTXexample}


\subsection{Avec thmtools}

A l'aide des exemples de la documentation de \href{http://mirror.ctan.org/macros/latex/exptl/thmtools/thmtools.pdf}{thmtools}, on peut notamment configurer la manière dont fonctionnent le compteur, les références, la couleur de fond, le style de caractères, l'indentation\dots


Il est relativement facile de personnaliser un style de paragraphe spécial, par exemple, pour lui ajouter un fond :

\begin{LTXexample}[pos=o]
    \begin{boxedDefinition}[Exemple]
        Début de la définition
    \end{boxedDefinition}
\end{LTXexample}

Pour créer ce fond, il suffit de créer un nouveau style, par exemple, ici dans le préambule :
\begin{minted}[autogobble]{latex}
  \declaretheoremstyle[notefont=\bfseries,parent=chapter,shaded={bgcolor=yellow}]{exempleStyle}
  \declaretheorem[style=exempleStyle,name=Définition]{boxedDefinition}
\end{minted}



\subsection{Avec mdframed}

L'environnement \mintinline{latex}{mdframed} offre des possibilités simples de faire des encadrés de paragraphes spéciaux, même sur plusieurs pages. On peut ainsi redéfinir des paragraphes spéciaux pour qu'ils aient une allure moins classique.

L'exemple ci-dessous est réalisé avec les options suivantes :

\begin{minted}{latex}
\mdfdefinestyle{exampledefault}{outerlinewidth=5pt,innerlinewidth=0pt,
outerlinecolor=red,roundcorner=5pt}  
\end{minted}

\global\mdfdefinestyle{exampledefault}{%
	outerlinewidth=5pt,	innerlinewidth=0pt, outerlinecolor=red,	roundcorner=5pt
}

\begin{mdframed}[style=exampledefault]
	\lipsum[1-8]
\end{mdframed}