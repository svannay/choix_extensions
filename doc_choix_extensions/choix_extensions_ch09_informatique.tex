% !TeX encoding = UTF-8
% !TeX root = choix_extensions.tex
\chapter{Informatique}





\section{Raccourcis prédéfinis}

Raccourci général pour le binaire :

\begin{LTXexample}[pos=o,width=.4]
	$100 \neq \binaire{100}$
\end{LTXexample}





\section{Colorisation syntaxique}

On peut importer ou copier du code source dans \LaTeX \ et lui laisser gérer la colorisation syntaxique et la présentation correcte du code. Pour cela, il existe plusieurs librairies, dont \incmd{minted}. Elle fait appelle à la librairie Python \incmd{Pygments} pour effectuer la mise en forme et l'insérer dans le document final. Voir \href{https://pygments.org/}{https://pygments.org/} pour se faire une idée de ce que peut faire Pygments.

Pour que \incmd{minted} puisse fonctionner, il faut que votre ordinateur dispose d'une distribution Python avec la librairie Pygments.



\subsection{Installation complémentaire}

Le plus simple est d'installer la version Open Source d'Anaconda qui contient tout ce qu'il faut\dots 
\begin{center}
	\href{https://www.anaconda.com/products/individual}{https://www.anaconda.com/products/individualb}
\end{center}

\remarque*{
	Si Python déjà installé, mais qu'il manque Pygments, il est toujours possible de l'installer après coup. Il faut ouvrir un terminal (pas celui de Python, mais bien celui de l'ordinateur) et y taper la commande
	\begin{center}
		\incmd{pip3 install Pygments}
	\end{center}
}



\subsection{Utilisation}

Pour accéder aux raccourcis spécifiques à l'informatique, il faut appeler \incmd{preambule_college.sty} avec l'option \incmd{informatique} :
\begin{center}
	\mintinline{latex}{\usepackage[informatique]{styles/preambule_college}}
\end{center}



\subsection{Raccourcis}

Pour faire de la colorisation syntaxique directement dans la ligne dans laquelle on écrit :

\begin{LTXexample}[pos=o,width=.4]
En bash : \newline
\incmd{grep 'LaTeX' *.tex}\\[.5ex]
En html : \newline
\incmd{<a href="#">anchor</a>}\\[.5ex]
En css : \newline
\incmd{h1{color:red}}\\[.5ex]
En js : \newline
\injs{var x = 12}\\[.5ex]
En java : \newline
\injava{for(int i=0;i<10;i++)}\\[.5ex]
En php : \newline
\inphp{if (isset($_POST['nom']))}\\[.5ex]
En sql : \newline
\insql{select * from table `nom`}\\[.5ex]
\end{LTXexample}

Avec l'environnement \mintinline{latex}{\begin{minted}{code}contenu...\end{minted}}. Par exemple, ici pour du code MySQL.
 
\begin{minted}{mysql}
CREATE TABLE IF NOT EXISTS tblproduit (
produit_ID SMALLINT NOT NULL
,  produit_description VARCHAR(80)
,  produit_typeReliure VARCHAR(8)
,  produit_prixUnitaire SMALLINT
,  PRIMARY KEY (produit_ID)
)  ENGINE=InnoDB ;
\end{minted}

Il y a évidemment plein d'options de configuration (cf. \href{https://www.ctan.org/pkg/minted}{minted} et \href{http://pygments.org/}{Pygments}). On peut les appeler depuis le préambule avec \mintinline{latex}{\setmintedinline{key=value}}.
