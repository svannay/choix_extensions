% !TeX encoding = UTF-8
% !TeX root = choix_extensions.tex
\chapter{Tableaux}





\section{Aérer les tableaux}



\subsection{Augmenter la hauteur de ligne}

Pour augmenter la hauteur des cellules, uniquement vers le haut, utiliser \mintinline{latex}{extrarowheight{...}}. Remarque : en mettant le \mintinline{latex}{extrarowheight{...}} dans un environnement \mintinline{latex}{center}, cela en limite l'effet à ce tableau.

\begin{LTXexample}[pos=o,width=.3\linewidth]
\begin{center}
  \setlength\extrarowheight{12pt}
  \begin{tabular}{|p{5ex}|l|}
    \hline
    P1 & $a^ma^n = a^{m + n}$ \\
    \hline
    P2 & $\left( {a^m} \right)^n = a^{mn}$ \\
    \hline
  \end{tabular}
\end{center}
\end{LTXexample}



\subsection{Augmenter la hauteur de ligne et centrer verticalement}

Pour augmenter la taille des cellules de manière centrée : insérer une réglure verticale de largeur nulle et décalée verticalement (p. ex. \mintinline{latex}{\rule[-2ex]{0pt}{5.5ex}}) :

\begin{LTXexample}[pos=o,width=.3\linewidth]
\begin{center}
  \begin{tabular}{|>{\rule[-2ex]{0pt}{5.5ex}}p{5ex}|c|}
    \hline
    P1 & $a^ma^n = a^{m + n}$ \\
    \hline
    P2 & $\left( {a^m} \right)^n = a^{mn}$ \\
    \hline 
  \end{tabular} 
\end{center}
\end{LTXexample}

C'est la version que propose l'\incmd{assistant tableau} de TeXstudio.





\section{Centrage horizontal et vertical}

Sur une ligne du tableau, pour faire aligner verticalement deux cellules : utiliser le type de colonne \mintinline{latex}{m{taille}}. Pour faire simultanément un centrage horizontal : rajouter un \mintinline{latex}{\centering} dans chaque cellule (en utilisant l'option de colonne \mintinline{latex}{>{}}).

\erreurCourante*{
	Il n'est plus possible alors d'utiliser \mintinline{latex}{\\} pour terminer une ligne, il faut utiliser explicitement \mintinline{latex}{\tabularnewline}.
}

\begin{LTXexample}[pos=o,width=.25\linewidth]
\begin{center}
  \begin{tabular}{|>{\centering}m{1cm}|>{\centering}m{1cm}|}
    \hline 
    {\Huge $\updownarrow$} & Flèche \tabularnewline
    \hline
    Flèche & {\Huge $\updownarrow$} \tabularnewline 
    \hline
  \end{tabular} 
\end{center}
\end{LTXexample}





\section{Tableaux mathématiques}
\label{sec:tableauxMaths}

Pour faire des tableaux en mode mathématique, utiliser l'environnement \mintinline{latex}{IEEEeqnarray} (voir \ref{sec:alignementEquations}). Comme alternative, il y a aussi \mintinline{latex}{array}. On peut en ajuster la hauteur de ligne avec \mintinline{latex}{extrarowheight} et l'espace inter colonne avec \mintinline{latex}{\arraycolsep} comme dans l'exemple ci-dessous :
\begin{LTXexample}[pos=o]
\[
  \setlength\extrarowheight{4pt}
  \setlength{\arraycolsep}{0pt}
  \begin{array}{rrrrr<{~}|>{~}l}
    4x^4 & -2x^3 & +0x^2 & +5x & -7 \hspace{0.3cm} & 2x^2+0x-3\\
    \cline{6-6}
    -4x^4 & -0x^3 & +6x^2 & & & 2x^2-x+3\\
    \cline{1-3}
    & -2x^3 & +6x^2 & +5x & & \\
    & +2x^3 & +0x^2 & -3x & & \\
    \cline{2-4}
    & & 6x^2 & +2x & -7 &\\
    & & -6x^2 & -0x & +9 &\\
    \cline{3-5}
    & & & 2x & +2 & \\
  \end{array}
\]
\end{LTXexample}

Noter aussi l'utilisation de la commande \mintinline{latex}{\cline} pour la réalisation de filets horizontaux sur une seule partie du tableau.





\section{Pense-bête}

Déclarer directement plusieurs colonnes de même type, fusion de cellules, filet partiel : 
\begin{LTXexample}[pos=o,width=.4]
\setlength\extrarowheight{1ex}
\begin{tabular}{|*{10}{c|}}
  \hline
  1 & 2 & 3 & 4 & 5 & 6 & 7 & 8 & 9 & 10 \\
  \cline{1-5}
  \multicolumn{5}{|c|}{Cinq colonnes} & \multicolumn{5}{c|}{Cinq autres} \\
  \cline{0-4}
  1 & 2 & 3 & 4 & 5 & 6 & 7 & 8 & 9 & 10 \\
  \hline
  1 & 2 & 3 & \multirow{2}{*}[2ex]{4} & 5 & 6 & 7 & 8 & 9 & 10 \\
  \cline{1-3} \cline{5-10}
  1 & 2 & 3 &  & 5 & 6 & 7 & 8 & 9 & 10 \\
  \hline
  1 & 2 & 3 & 4 & 5 & 6 & 7 & 8 & 9 & 10 \\
  \hline
  1 & 2 & 3 &
    \multicolumn{3}{c|}{\multirow{2}*{4 \ 5 \ 6}}
    & 7 & 8 & 9 & 10 \\
  \cline{1-3} \cline{7-10}
  1 & 2 & 3 &
    \multicolumn{3}{c|}{} & 7 & 8 & 9 & 10 \\
  \hline
  1 & 2 & 3 & 4 & 5 & 6 & 7 & 8 & 9 & 10 \\
  \hline
\end{tabular}
\end{LTXexample}





\section{Tables professionnelles}

Avec l'extension \href{https://www.ctan.org/pkg/booktabs}{booktabs.sty}, on peut faire des tables correspondant aux standards typographiques professionnels :

\begin{LTXexample}
\begin{tabular}{@{}llc@{}} \toprule
    $P$ & $Q$ & $P \;et\; Q$ \\ \midrule
    $\cV$ & $\cV$ & $\cV$ \\
    $\cV$ & $\cF$ & $\cF$ \\
    $\cF$ & $\cV$ & $\cF$ \\
    $\cF$ & $\cF$ & $\cF$ \\ \bottomrule
\end{tabular}
\end{LTXexample}
